\documentclass{article}
\usepackage[utf8]{inputenc}
\usepackage{hyperref}

\title{Carbon Charts: Proof of Concept Contract}
\author{
    McMaster University\\
    Group 18 \\
}

\date{\today}

\begin{document}

\maketitle
%\section{Report Details}

%You will be graded on your choice of what you decided to demo, and whether or not your demo convinces us that you can deal with the problem(s) you chose to overcome. The grade will be out of 20. (It counts 5\% towards your final grade.) Rough guide will be 4 marks for the choice of what you demo, 10 marks for the demo itself, and 6 marks will be allocated to the submitted "contract" of what you will guarantee to demo in the final presentation. I will open a dropbox for the contract.  Due Date will be Sunday Nov 18 at 23:59 and End Date (dropbox closes) will be Friday Nov 23 at 23:59. The assignment dropbox will list the entire grade for the Proof Of Concept, not just the "contract".

%The Proof of Concept "Contract" should be submitted as a pdf.  It should be brief. An intro (1 mark), 3-5 bullet points (3 marks) and a conclusion (2 marks). You should not list stretch goals for your project, just the items you will be able to demo in your final presentations.

%The intro should provide a very brief overview of your project. The bullet points should describe scenarios that you will successfully present. When you choose these scenarios, please: 

%i) Choose scenarios that make it clear that in order to demo them, you will have had to substantially complete your project (ie do not choose such easy demo scenarios that you could accomplish them without really completing your project); 

%ii) At the same time, do not promise to complete stretch goals. It will be great if you do complete stretch goals and demo them successfully, just do not promise them in this contract; and

%iii) in a conclusion section, indicate to us why you think that successful demos of your contract scenarios imply that you will have substantially completed your project.

%Alan


\section{Introduction}
Carbon Charts is an open-source JavaScript data visualization library that will be used as a functional component of IBM's Carbon Design System.\\

The proof of concept will demonstrate:
\begin{itemize}
    \item Data visualization using Carbon Charts.
    \item All chart types described in the requirements document. (Line, Area, Bubble, Pie, Donut, Stacked Bar, Grouped Bar).
    \item Real-time recording of various sensory data. This data was generated by a simulation in our demo but will be fulfilled with a hardware component in the proof-of-concept.
    \item Support in all Tier A browsers will be demonstrated (latest versions of Chrome, Firefox, Opera and Safari).
\end{itemize}

Demo: \href{https://yj8j52vl7z.codesandbox.io}{https://yj8j52vl7z.codesandbox.io}

\section{Conclusion}
Sucessful demos of our proof of concept contract indicates that we have substantially completed our project. This is beacause the proof-of-concept includes views from the stakeholders, application developers and users. Due to the open source nature of this library, application developers must be able to easily extend existing functionaility for their indended use. In our proof-of-concept we demonstrated this by simulating real-time data while concurrently visualizing the data via various charts provided through the Carbon Charts library. Application users will use the charts generated by the charting library to visually interpret data, recognize patterns and derive insights.

\end{document}
